\documentclass[acmtocl]%{acmtrans2m}%
{boviktrans}

\sloppypar

\usepackage{dashrule}
\usepackage{amssymb}
\usepackage{amsmath}
%\usepackage{amsthm}
\usepackage{stmaryrd}

\newcommand{\seq}[3]{{#1};{#2} \longrightarrow {#3} \mathstrut}

\newcommand{\urfoc}[3]{{#1};{#2} \longrightarrow [{#3}] \mathstrut}
\newcommand{\ulfoc}[4]{{#1};{#2} \,[#3] \longrightarrow {#4} \mathstrut}
\newcommand{\uinv}[4]{{#1};{#2};{#3} \longrightarrow {#4} \mathstrut}

\newcommand{\stableR}[1]{{#1}\,\mathit{stable_R} \mathstrut}
\newcommand{\stableL}[1]{{#1}\,\mathit{stable_L} \mathstrut}
\newcommand{\one}{{\bf 1}}

\markboth{Simmons \& Murphy}{A modest proposal for the purity of programming}

\title{A modest proposal for the purity of programming}
\author{Robert J. Simmons \and Tom Murphy VII}

\begin{abstract}
  Terrible ideas permeate the world of programming languages, and the
  harm that these ideas do is lasting. Academic research is intended
  to ameloriate some of this harm, but the connection between academic
  PL research and industry grows ever more tenuous. This harms both
  realms. Terrible ideas continue to hold back the benefits that
  computer science and software enginnering seek to bring to the
  world.  On the other side, we should note the demoralizing effect of
  this tendency on academic PL research: why work on ``practical''
  research if you will be universally ignored?

  The more principled among us might suggest that we continue to
  generate the best ideas and see what happens, that the right ideas
  will win in time. The time for such velvet-glove approaches is long
  past. This is a war, and it is time to use all the resources at our
  disposal. We propose a method and apparatus to save the world from
  itself.
\end{abstract}

\category{F.4.1}{Theory of Computation}{Mathematical Logic}[Proof theory]
            
\terms{} 
            
\keywords{}

\begin{document}

\maketitle

\section{Introduction}

Programming languages are fundamentally structured expressions of
human thought; they allow mere mortal humans like you and I to wield
the power of the fantastically complex and powerful computing devices
that live in our phones, coffee machines, and MacBooks. Like most
other human enterprises, programming languages are subject to fad and
whim and fashion. {\it Unlike} most other human enterprises, the
damage done by ill-considered ideas can be exceptionally lasting if
the languages incorporating these ideas are used to build big
important systems. If you dislike covariant generics, you should mourn
that we will seriously be stuck with Java's covariant arrays until the
heat death of the universe. If you {\it do} like covariant generics,
you are wrong, but nevertheless you should mourn that we will
seriously be stuck with Java's invariant generics until the heat death
of the universe.

The research community on programming languages, imperfect as it is,
has usually seen the worst of these disasters coming in
advance. However, in the diffuse and surreally contentions community
of programming languages research, attempts to understand errors and
contain their worst damage of bad ideas seem tend only to solemnize
their status as ``well-understood features of modern programming
languages,'' ensuring their inclusion in the next great disaster.

We propose a radically different strategy: we will save the future
from our follies of today by setting our countenances towards
discovering tomorrow's bad ideas and tasteless fads {\it first}. From a
pure research perspective, this is the ultimate folly, not that that's
ever stopped us before
\cite{sigbovik07,sigbovik08,sigbovik09,sigbovik10,sigbovik11}. But the
stakes are too high to keep our hands clean. It's time to bring out
the big guns, and put our bad ideas to work in the (regulatory)
marketplace.

\subsection{Primer on patent law}

Patent law in the United States traces back to Article 1, Section 8 of
the United States Constitution, which gives Congress the power ``to
promote the Progress of Science and useful Arts, by securing for
limited Times to Authors and Inventors the exclusive Right to their
respective Writings and Discoveries.'' This is the same portion of the
constitution that copyright law stems from; however, in their core
functionality patent law and copyright law are quite
different. Patents are particularly unusual in the degree of
exclusivity they give: a patent gives the holder a {\it monopoly} over
the exercise of their patent for a period of time. Under current U.S.
law, this period of time extends 20 years from the date when the
patent was officially filed.

One view of the patent system is that it is a deal the public makes
with an inventor. Patents must provide enough information so that a
skilled person can carry out the claimed invention -- this is called
the {\it sufficiency of disclosure} requirement.  So, a patent gives
you 20 years when you have absolute control over who is allowed to use
your invention, but this awesome power comes at a cost -- you
basically guarantee that, in 20 years, everybody is going to be able
to use your invention. Furthermore, because patents are published
right away, you give everybody else a head start on innovating further
based on your ideas -- and if someone else comes with an innovation
that's awesome enough, they can get a brand new patent that {\it you}
can't use. And Science advances!  If you don't want to give your good
ideas to anybody else right away and/or if you think you could keep
your awesome secret a secret for more than 20 years, then it's to your
advantage as an inventor {\it not} to make this trade, and have your
invention be a trade secret rather than a patented invention.

That, at least, is the civics book lesson for how patents work, but
the patent system has had some difficulties coping with the complex
nature technological innovation in the modern world. We will briefly
describe two phenomena \cite{dododododod} that have arisen around
patent law in the context of modern information technology: the
standardization of {\it reasonable and non-discriminatory} licensing
terms, and the scourge of {\it patent trolls}. Both of these phenomena
\cite{dododododod} are relevant to our method and apparatus for saving
the world, {\tt The /dev/urand/ Foundation}, described in Section 3.

\subsubsection{Reasonable and non-discriminatory terms}
\label{sec:rand}

Standards bodies have to work around the fact that many standards
inevitably are covered by patents. Standards-setting organizations
allow for patented technologies to be used in standards, but require
that the patent holder provide {\bf R}easonable {\bf a}nd {\bf
  N}on-{\bf D}iscriminatory licensing options, or RAND. This term
isn't terribly well defined, but the intent is to ensure that anyone
can implement the standard, by paying a fair licensing fee to the
patent holder until the patent expires, and is generally good for
technology.

To see why this is important, consider a standards body with
representatives from 57 companies all working on a new standard for
Carrier Pigeon Message Formatting modernizing RFC 1149. BBN Labs has a
new patent on avian foot massage technology using cardboard
\cite{ebert}. Without revealing this fact to the consortium, BBN Labs
influences the standards body to incorporate the requirement for
post-packet-delivery corrugated massage as a quality assurance
mechanism. Then, after waiting for the standard to be incorporated
into every soda machine in America, BBN Labs can demand arbitrary fees
from the users and distributors of their massage-enhanced pigeon
packet technology. If people refuse to pay, they can be legally barred
from using the (standards-backed) technology they have already bought
and paid for.

RAND licensing is aimed at avoiding this scenario. When
RAND terms are required, then BBN Labs still has something to gain
from the adoption of their patented technology in the standard, but
they have less to gain from concealing this from their standards-body
partners. RAND is not a solution to patent-encumbered standards,
merely a way to make them work in the real world. But, critically,
RAND licensing is not itself a fundamental part of patent law -- if
BBN Labs are {\it not} a part of the standards committee, then they can
still wield their patents despite the standards committee members
being bound by RAND terms when it comes to their own patented
inventions. 

\subsubsection{Patent trolls}

The interrelated nature of technological innovation, and the frequency
with which fundamental ideas spring into existence in multiple places
at the same time, has led to the prevalence of {\it patent trolls}.
The term is generally defined as people and companies that buy lots of
patents, wait for people to start using the ideas contained therein
naturally, and then extort the maximum rents possible from the users.

Patent trolls generally wield two types of patent - the specific
patent that is nearly certain to get rediscovered in time, and the
general, overly-broad patent that basically apply to everything ever,
like the people who freakin' patented {\it linked lists} in freakin'
2004 \cite{want} or the patent trolls currently suing all your cell
phone makers because they let you select emoticons from a list
\cite{smile}.  The latter form of troll patent is often cited as
evidence for the necessity of patent reform, as, despite the fact that
these patents are usually unenforceable, the mere threat of patent
litigation can be used to extract rents from other innovators and have
a chilling effect on innovation overall.

\subsection{The chillaxing effect}

Our method and apparatus was inspired by some person on Twitter
\cite{tweet}, though we stress that this does not count as prior art.
As the random Twitter-person observes, it's a good thing that the
fundamental good ideas in computer science aren't covered by patents,
because they are useful for helping droves of other computer
scientists do their work.  But what if the {\it bad} ideas in computer
science were covered by patents? It might prevent droves of other
computer scientists from doing their work. This {\it chillaxing
  effect}, effectively wielded, could simultaneously get the attention
of the largest corporate players in programming languages and software
engineering and refocus the efforts of academic research in directions
that have not been meticulously chillaxed.

I mean, we tried to come up with a bad idea a couple of years ago
\cite{sigbovik10}, and then we learned at the ICFP in Baltimore that
Milner had actually come up with the {\it same bad idea} in one of the
early ML implementations, before replacing it with the less dumb idea
later on. Milner could have {\it patented} this bad idea when he came
up with the better idea, thus giving the forces of sanity with a
powerful tool against the next person who tries to implement the bad
idea and stop there.

\section{Examples}


Coming up with bad ideas for programming languages is very easy. The
challenge is to come up with ideas that are broad enough to cover many
possible instantiations of the idea, specific enough to be patentable,
and likely to be encountered in real upstart languages, where they can
be stamped out. As usual in science, our approach is stochastic; we
simply generate as many patents as we can. Many patents will never be
useful in the fight against bad programming languages, but these cause
no harm.\!\footnote{Informal studies in CMU's Principles of
  Programming group have shown preliminary evidence that bad
  programming languages can actually cause physical harm among those
  that have established taste and predisposition to logic. Observed
  effects include facepalms and grimaces, nausea, fatigue, emotional
  lability (uncontrolled weeping or laughing), Bobface (first identified
  by William J.~Lovas), and dry mouth. Other harm is more direct, such
  as lacerations or bruises by being throttled by academic advisors.
  If this proves to be a problem, simple safety measures such as biohazard
  suits, coordinate-transform and other reversible mind-encryption systems,
  or simply employing the inadverse, may be used.}
As a demonstration, this section contains a list of bad programming language
ideas that we came up with, no sweat. Cringe away:

\begin{enumerate}
\item The input programs are written as recipes
\item You just give examples of what a function should do in certain circumstances, and when it encounters an input that is not specified it\ldots
  \begin{enumerate}
    \item \ldots linearly interpolates between known answers
    \item \ldots uses genetic programming to come up with a short program that satisfies your constraints and also works on this input
    \item \ldots pauses and waits for the programmer to finish the program
    \item \ldots asks the user what the answer should be, adding it to the database
    \item \ldots searches for code on the internet that meets the example-based specs, and prompts the programmer or user as to which one should be used
    \item \ldots
  \end{enumerate}
\item Input programs are written in musical notation
\item Input programs are graphical diagrams written in UML, XML, flow charts, as maps, circuits, or two-dimensional ASCII
\item Programs are written in three-dimensional layered text, perhaps in different colors and with alpha channels, to specify interleaved threads
\item Every program is a substring of the {\it lorem ipsum} text
\item Everything in the language is just a\ldots
   \begin{enumerate}
     \item \ldots string literal, including keywords
     \item \ldots capital {\it i} or lowercase {\it L}
     \item \ldots continuously differentiable probability density function
     \item \ldots hash table mapping hash tables to hash tables
     \item \ldots $n$-tuple
     \item \ldots finite permutation
     \item \ldots 7-bit integer
     \item \ldots coercion
     \item \ldots rule
     \item \ldots exception, except exceptions; those are normal
     \item \ldots arbitrary-precision rational number
     \item \ldots priority queue, Fibonacci heap, b-tree, pixel, regular expression, presheaf, commutative diagram, metaphor, monad
     \item \ldots MP3
     \item \ldots SMS
     \item \ldots mutex
     \item \ldots non-uniform rational b-spline
   \end{enumerate}
\item Programming languages for children or the elderly
\item Programming languages based on telling stories
\item Programming languages based on architecture, org charts, HTML, CSS, military strategy, or airplanes
\item Instead of stack-based control-flow, use queue-based, tree-based, dataflow-network-based
\item 4/3 CPS
\item Every value is represented as the 256-bit content hash; elimination forms are distributed hashtable lookups; revision control is built into the concrete syntax of the language
\item Unification always succeeds, forking the program with each of the two expressions to be unified substituted in that position; only if both fail does unification fail
\item Realize every program you wish to write as actually the test case for a metaprogram that generates the program
\item Language with only 20 keywords, one for each of the SPEC benchmarks
\item Language with only one keyword, whose semantics implements a compiler for the language itself
\item {\tt call-ac}, call with all continuations
\item Second-class data: All data must be top-level global declarations, and can't change. Functions are first-class.
\item Gesture-based concrete syntax
\item Programs are realized as dashboards with knobs, buttons, and cable connections between them
\item No matter what, the program keeps going, attempting to repair itself and keep trying actions that fail
\item Programs are abstract geometric shapes
\item Type has type \ldots
  \begin{enumerate}
    \item type
    \item int
    \item kind
    \item type $\rightarrow$ type
    \item object
    \item null
  \end{enumerate}
\item To protect against the problem where sometimes someone called a function with an empty string, ``emptyable'' types, which include all values of the type except the ``empty'' one (\verb+""+, 0, NaN, 0.0, nil, \verb+{}+, false, etc.).
\item To work around the global {\tt errno} problem, every value of a type includes the possibility of integers standing for an error code
\item Lazy natural (co-)numbers, where the output of a numeric program is only a lower bound that may get higher as it continues computing
\item A language where the compiler is integrated into the language as a feature, which takes first class source code to first class compiled binaries, within the language
\item There's a global registry, in the world, and whenever a function returns, you check to see if any function in the world has registered a hook to process it
\end{enumerate}

You see how easy this is? If you are a programming language expert you
might even have thought of some languages that already use these
ideas. If so, {\it this is all the more reason to support our
  foundation}, because had we started earlier, we could have saved the
world some trouble!

It is worthwhile to try to acquire patents that are very broad, since
these can be used to attack almost any language, even one with
unanticipated bad ideas. For example:


% XXX minipage?
\subsection{Method and apparatus for attaching state to an object}

This patent describes a method and apparatus for attaching state to
objects in computer programs. The invention consists of a symbolic
program running in computer memory and an object (which may be a value,
hash table, list, function, binary data, array, vector, $n$-tuple,
presheaf, source file, class, run-time exception, finite or infinite
tape, or isomorphic representation). The claims are as follows:

1.\quad A system for attaching state to the said object \\

2.\quad The method of claim 1 where the state is binary data \\

3.\quad The method of claim 1 where the state is an assertion about the behavior of the object \\

4.\quad The method of claim 1 where the state is itself an object \\

5.\quad The method of claim 4 where the state is the same object, or some property therein \\

6.\quad The methods of claims 1--5 where the apparatus of attachment is reference \\
\qquad 6.a.\quad Reference by pointer \\
\qquad 6.b.\quad Reference by index \\
\qquad 6.c.\quad Reference by symbolic identifier \\
\qquad 6.c.\quad Representations isomorphic to those in claims 6.a.--6.c. \\

7.\quad The methods of claims 1--5 where the apparatus of attachment is containment \\

{\it (etc.)}

\subsection{Method and apparatus for determining the control flow of programs}

This patent describes a method and apparatus for determining the
control flow in computer programs. The invention consists of a
symbolic program running in computer memory, with a notion of {\it
  current} and {\it next} state (which may be an instruction pointer,
index, expression to evaluate, continuation, value, covalue, stack,
queue, execution context, thread or thread pool, process image,
cursor, tape head, phonograph stylus, or isomorphic
representation). The claims are as follows:

1.\quad A system for determining the next state from the previous state \\

2.\quad The method of claim 1, where the determination includes the contents of the program's memory \\

3.\quad The method of claim 1, where the determination includes the current state \\

4.\quad The method of claim 1, where the determination includes external inputs \\

5.\quad The method of claim 1, where the determination includes nondeterministic factors \\

6.\quad The method of claim 1, where the determination is fixed ahead of time \\

{\it (etc.)}

It is hard to imagine any programming language that would not be
covered by both of these patents. Many perfectly sensible languages
would be impacted as well, but this is not a problem: We can choose to
license the patent to languages that we judge to be tasteful, perhaps
imposing additional contractual restrictions on the licensees, even
regarding things not covered by our patent pool. We can easily
develop hundreds of such applications and again use the stochastic
method to ensure a high likelihood of having one granted.

\section{The /dev/urand Foundation}

The patents shall be administered by a new non-profit foundation,
known as {\tt The /dev/urand Foundation}. The organization is named for the
RAND concept of patent licensing described in Section~\ref{sec:rand}. 
The /dev/urand Foundation differs in that its licensing is {\bf
  U}n{\bf r}easonable {\bf a}nd {\bf N}ot-not {\bf D}iscriminatory: We
do not offer licenses for any amount of money or other consideration.
Our patents on bad ideas will simply never be licensed, enjoining
anyone from using those ideas for the duration of the patents. 
Our over-broad patents will be licensed
in a blatantly discriminatory fashion, only to languages that we think
are tasteful. 
We might even withhold licenses when an individual has
done something that we just don't like, like has a name that's hard to
spell, or didn't accept one of our papers to a prestigious
conference. This is totally legal.

We are going up against some of the biggest companies in the world,
such as Larry Wall and Guido van Rossum, however, and so we anticipate
that specific patents on bad ideas will be more powerful than 
potentially indefensible over-broad patents. However, there is certainly
a role for both kinds of patent trolling.
The point is to strike fear into the hearts of would-be hobbyists and
academics, with the expectation that this would be generally {\it bad}
for the programming language ecosystem, which we can all agree is
% Rob: some kind of pun here instead?
pretty much up shit's creek without a paddle.

Lawsuits: coming to a workshop on reinventing the wheel near you.

\section{Conclusion}

We have presented a plan for saving programming from the scourge of
clumsy innovation. Surely the plan is distasteful, and perhaps you
think the world would be a better place if the negative effects of
out-of-control patent law -- whether they chilling (bad) or chillaxing
(awesome) -- were curtailed with the limitation or elimination of
software patents. Maybe so! But the problem with refusing to let the
ends justify the means is that when you do that {\it the other team
  ends up with more means}. And we refuse to settle for average.

We will fight fire with whatever
firefighting means we can get our fricking hands on. 
The future of programming depends on
it!

\begin{thebibliography}{1}

\bibitem{sigbovik11}
Blum, Benjamin (ed).
\newblock ``Proceedings of SIGBOVIK 2011.''
\newblock ACH Press.
\newblock Pittsburgh, Pennsylvania.
\newblock 2011.

\bibitem{dododododod}
Do do dododo. 
\newblock ``Phenomena.''
\newblock Do do do do. 
\newblock ``Phenomena.'' 
\newblock Do do dododo dododo dododo dododododo do do do do do.

\bibitem{ebert}
Ebert, Michael A.
\newblock ``Corrugated recreational device for pets.''
\newblock Patent Application 11/757,456.
\newblock Filed June 4, 2007.

\bibitem{tweet}
Gorman, Jason. 
\newblock (jasongorman).
\newblock ``We should think ourselves very lucky that Alan Turing didn't patent ``a single machine which can be used to compute any computable sequence''''
\newblock March 13, 2012.
\newblock Tweet.

\bibitem{sigbovik09}
Jones, Laurie A (ed).
\newblock ``Proceedings of SIGBOVIK 2009.''
\newblock ACH Press.
\newblock Pittsburgh, Pennsylvania.
\newblock 2009.

\bibitem{sigbovik07}
Leffert, Akiva and Jason Reed (eds).
\newblock ``Proceedings of SIGBOVIK 2007.''
\newblock ACH Press.
\newblock Pittsburgh, Pennsylvania.
\newblock 2007.

\bibitem{smile}
Nelson, Jonathan O.
\newblock ``Emoticon input method and apparatus.''
\newblock Patent 7,167,731 B2.
\newblock Granted January 23, 2007.

\bibitem{sigbovik10}
Martens, Chris (ed).
\newblock ``Proceedings of SIGBOVIK 2010.''
\newblock ACH Press.
\newblock Pittsburgh, Pennsylvania.
\newblock 2010.

\bibitem{sigbovik08}
Simmons, Robert J. (ed).
\newblock ``Proceedings of SIGBOVIK 2008.''
\newblock ACH Press.
\newblock Pittsburgh, Pennsylvania.
\newblock 2008.

\bibitem{simmons10}
Simmons, Robert J., Nels E. Beckman, and Dr. Tom Murphy VII, Ph.D.
\newblock ``Functional Perl: Programming with Recursion Schemes in Python.''
\newblock In {\it SIGBOVIK 2010}.

\bibitem{want}
Wang, Ming-Jen.
\newblock ``Linked list.''
\newblock Patent 7,028,023.
\newblock Granted April 11, 2006.


\end{thebibliography}

\end{document}
