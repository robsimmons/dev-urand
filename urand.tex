\documentclass[acmtocl]%{acmtrans2m}%
{boviktrans}

\usepackage{dashrule}
\usepackage{amssymb}
\usepackage{amsmath}
%\usepackage{amsthm}
\usepackage{stmaryrd}

\newcommand{\seq}[3]{{#1};{#2} \longrightarrow {#3} \mathstrut}

\newcommand{\urfoc}[3]{{#1};{#2} \longrightarrow [{#3}] \mathstrut}
\newcommand{\ulfoc}[4]{{#1};{#2} \,[#3] \longrightarrow {#4} \mathstrut}
\newcommand{\uinv}[4]{{#1};{#2};{#3} \longrightarrow {#4} \mathstrut}

\newcommand{\stableR}[1]{{#1}\,\mathit{stable_R} \mathstrut}
\newcommand{\stableL}[1]{{#1}\,\mathit{stable_L} \mathstrut}
\newcommand{\one}{{\bf 1}}

\markboth{Robert J. Simmons, {\it et al.}}{A modest proposal for the purity of programming}

\title{A modest proposal for the purity of programming}
\author{Robert J. Simmons \and Tom Murphy VII}

\begin{abstract}
  Terrible ideas permeate the world of programming languages, and the
  harm that these ideas do is lasting. Academic research is intended
  to ameloriate some of this harm, but the connection between academic
  PL research and industry grows ever more tenuous. This harms both
  realms. Terrible ideas continue to hold back the benefits that
  computer science and software enginnering seek to bring to the
  world.  On the other side, we should note the demoralizing effect of
  this tendency on academic PL research: why work on ``practical''
  research if you will be universally ignored?

  The more principled among us might suggest that we continue to
  generate the best ideas and see what happens, that the right ideas
  will win in time. The time for such velvet-glove approaches is long
  past. This is a war, and it is time to use all the resources at our
  disposal. We propose a method and apparatus to save the world from
  itself.
\end{abstract}

\category{F.4.1}{Theory of Computation}{Mathematical Logic}[Proof theory]
            
\terms{} 
            
\keywords{}

\begin{document}

\maketitle

\section{Introduction}

Why are programming langauges important? Why are they broken? What is our approach to fixing it?

\subsection{Primer on patent law}

Patent law is 

How does patent law work, and how does it allow us to stomp out bullshit?

\subsubsection{Unreasonable and non-discriminatory terms}

Standards bodies require member organizations to adhere to the mostly
ill-defined requirement of using {\it reasonable and
  non-discriminatory terms}, RAND for short, for any and all patents
owned by those organizations that pertain to the standard.

To see why this is important, consider a standards body with
representatives from 57 companies all working on a new standard for
Carrier Pidgeon Message Formatting modernizing RFC 1149. BBN Labs has
a new patent on avian foot massage technology \cite{ebert}, and without
revealing this fact to the consortium they incorporate the requirement
to 

\subsection{Base coat on patent trolls}

- different levels of patent from rock solid to pure troll-threats

We are going up against some of the biggest companies in the world, such as Larry Wall and Guido van Rossum.

\subsubsection{The chillaxing effect}

\subsection{Shellac on lost opportunities}


Examples of history: SIGBOVIK/Milner idea


\section{Examples}

tom

\section{The /dev/urand Foundation}

tom

UnReasonable and Not-not Discriminatory

\section{Lawsuits are coming}

languages that we are coming after


\begin{thebibliography}{1}


\bibitem{ebert}
Ebert, Michael A.
\newblock ``Corrugated recreational device for pets.''
\newblock Patent Application 11/757,456.
\newblock Filed June 4, 2007.

\bibitem{simmons10}
Simmons, Robert J., Nels E. Beckman, and Dr. Tom Murphy VII, Ph.D.
\newblock ``Functional Perl: Programming with Recursion Schemes in Python.''
\newblock In {\it SIGBOVIK 2010}.

\end{thebibliography}

\end{document}
